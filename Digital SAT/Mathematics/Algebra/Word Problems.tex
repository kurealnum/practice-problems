\documentclass{article}
\usepackage{graphicx}
\usepackage{parskip}
\usepackage{hyperref}
\usepackage{amsmath}

\begin{document}
\section{Word Problems}

\begin{enumerate}
	\item {The weight, $w$, in ounces, of a glass soda bottle design with a side length of $l$ inches is given by the equation. By how many ounces does the weight of the bottle increase for every inch of increase in the side length, with the following equation (1)? \[w=3.1 + \frac{l}{2}\]

	      Hints:
	      \begin{enumerate}
		      \item{Can we change the variables in order to put this equation into a more familiar form, that being point-slope form?}
		      \item{What is the slope the equation? Remember that \(\frac{1}{2}*x\) is the same as \(\frac{x}{2}\)}
	      \end{enumerate}

	      Answer: 0.5}
	\item {Given the equation $C = 290+155h$: A team of painters is hired to paint the exterior of a museum. The total cost, $C$, in dollars, incurred by the museum for $h$ hours of painting is given by the equation. What is the best interpretation of $155$ in the equation (1)?

	      Hints:
	      \begin{enumerate}
		      \item {What is the slope of the graph?}
	      \end{enumerate}

	      Answer: Each additional hour of painting the exterior costs $\$155$}
	\item {
	      On Monday, Harry had $75\%$ as many toys as Teddy did. On Tuesday, Harry acquires $32$ more toys, and Teddy acquires $15\%$ as many toys as he had on Monday. If Harry and Teddy now have the same number of toys, how many toys did Harry have on Monday (1)?

	      Hints:
	      \begin{enumerate}
		      \item {How can you express how many toys Harry had on Monday, compared to Teddy?}
		      \item {How can you express how many toys Harry had on Tuesday, compared to Teddy?}
	      \end{enumerate}

	      Answer: 60
	      }
	\item {
	      At the beginning of the season, MacDonald had to remove $5$ orange trees from his farm. Each of the remaining trees produced $210$ oranges for a total harvest of $41790$ oranges. If $t$ is the initial number of trees on MacDonald's farm, what equation best describes the situation (1)?
	      Hints:
	      \begin{enumerate}
		      \item {How many oranges does each tree, $t$, produce?}
		      \item {How many oranges do all of his trees produce?}
		      \item{What do you need to subtract from the $t$? How can you express that algebraically?}
	      \end{enumerate}

	      Answer: $210(t-5)=41790$
	      }
	\item {
	      To rent a car for one week, a car rental company charges a $\$200$ base price as well as $\$0.45$ per mile. Jennifer will rent a vehicle at this company, but she has a $\$275$ budget. How many miles can Jennifer drive without exceeding her budget (1).

	      Hints:
	      \begin{enumerate}
		      \item{How can you express the amount the Jennifer should spend?}
		      \item{If you were to make a linear equation with the given information, what would the slope be? How about the $y$ intercept?}
		      \item{Remember to round down!}
	      \end{enumerate}

	      Answer: 166 miles
	      }
	\item {
	      A concert venue sold two types of tickets for an upcoming concert: reserved tickets and general admission tickets. Reserved tickets sold for $\$50.00$, and general admission tickets sold for $\$34.00$ each. If $1520$ tickets were sold for a total of $\$64000$, what is a system of equations that could be used to find the number of reserved tickets, $r$, and the number of general admission tickets $g$, that were sold (1)?

	      Answer:
	      \[
		      \begin{array}{l}
			      \begin{cases}
				      $50r+34g=64000$ & \\
				      $r+g=1520$
			      \end{cases}
		      \end{array}\]
	      }
	\item {Trevon is going to buy a coat and a hat. The coat costs $3$ times as much as the hat. He must spend less than $\$94$. Write an equation to express this situation (1).

	      Hints:
	      \begin{enumerate}
		      \item {This should be an inequality.}
		      \item {Don't forget to include the price of the hat.}
	      \end{enumerate}

	      Answer: \(4x < 94\)}
	\item {Cell phone provider A charges $\$60$ a month for $1$ gigabyte (GB) of data plus $\$0.05$ for each megabyte (MB) of overage data. Cell phone provider B charges $\$45$ a month for $1$ GB of data plus $\$0.10$ for each MB of overage data. Assuming there will be data overage, how many MB of overage data would make the cost of both data plans the same (1)?

		      Hints:
		      \begin{enumerate}
			      \item {What is overage data?}
			      \item {Make an equation for provider A and B.}
			      \item {What is the slope and y-intercept for each equation?}
			      \item {What is the solution in the context of the equation?}
		      \end{enumerate}

		      Answer: $300$
	      }


	\item {Fabio and Carlos play on a basketball team together. In the last game, Fabio had $7$ points less than $2$ times as many points as Carlos. Fabio scored $31$ points in the game. How many points did Carlos score in the last game (1)?

	      Hints:
	      \begin{enumerate}
		      \item {Try making a few equations with the given information. \(\text{Fabio}=F\text{ and Carlos}=C\)}
		      \item {One of the equations is \(F=31\)}
		      \item {Another equation could be \(F=2*C-7\)}
	      \end{enumerate}

	      Answer: 19
	      }

	\item {Marc is an aspiring music artist. He has a record contract that pays him a base rate $\$200$ a month and an additional $\$12$ for each album he sells. Last month he earned a total of $\$644$. If $a$ is the number of albums Marc sold last month, write an equation that best describes the situation (1).

	      Hints:
	      \begin{enumerate}
		      \item {What should one side of this equation be (a single value)?}
		      \item {Marc makes $a$ albums, and earns $\$12$ per album}
	      \end{enumerate}

	      Answer: \(200+12a=644\)
	      }
	\item {
	      Ricardio has two types of assignments for his class. The number of mini assignments, $m$, he has is $1$ fewer than twice the number of long assignments $l$. If he has $46$ assignments in total, create a system of equations that can be used to correctly solve for $m$ and $l$ (1).

	      Answer:
	      \[
		      \begin{array}{l}
			      \begin{cases}
				      m=2l-1 & \\
				      m+l=46
			      \end{cases}
		      \end{array}
	      \]
	      }
	\item {The equation \(5.5B+4R=28\) models the cost if Amit picks $B$ pounds of blueberries and $R$ pounds of raspberries at a farm where blueberries cost $\$5.50$ per pound a raspberries cost $\$4.00$ per pound. According to the equation, how much does Amit spend in total on both types of berries (1)?

		      Answer: $\$28.00$}
	\item {A rain barrel is being used during the dry season to water a garden with previously collected rainwater. Its equation is modeled by \( w=500-6.2t \). With a hose connected to its outlet, the equation approximates amount of rainwater in the barrel, $w$ in liters, after $t$ minutes of watering. What is the meaning of $6.2t$ in this equation (1)?

	      \begin{enumerate}
		      \item {It takes $6.2t$ minutes to empty the barrel}
		      \item{$6.2t$ minutes are needed to fill the barrel}
		      \item{$6.2t$ liters of rainwater is emptied after $t$ minutes}
		      \item{There are $6.2t$ liters in the barrel when filled for $t$ minutes}
	      \end{enumerate}

	      Hints:
	      \begin{enumerate}
		      \item{Define $6.2t$ without considering any of the possible answers. How can this definition be morphed to fit the needs of the problem?}
	      \end{enumerate}

	      Answer: C
	      }

	\item{It takes $40$ ink cartridges and $200$ pages to print a book, and it takes $30$ ink cartridges and $80$ pages to print a magazine. Sarah wants to print books and magazines with at most $300$ ink cartridges and $1200$ pages. Let $B$ denote the number of books she prints and $M$ the number of magazines she prints. Create a system of inequalities that describes this situation (1).

	      Hints:
	      \begin{enumerate}
		      \item{This is very similar to easier problems in this format, it just introduces another layer of complexity}
		      \item{How can you take the number of ink cartridges in both statements, and vice versa?}
	      \end{enumerate}

	      Answer:
	      \[
		      \begin{array}{l}
			      \begin{cases}
				      40B+30M \leq 300 & \\
				      200B+80M \leq 1200
			      \end{cases}
		      \end{array}
	      \]
	      }

	\item{Dew point temperature, in degrees Celsius, is defined as the temperature to which the air would have to cool in order to reach saturation. For a temperature of $35$ Celsius, an estimate of the dew point can be obtained by first subtracting $20$ Celsius from the temperature, and then adding $1$ degree for every increase of $5$ points in the relative humidity, $R$. If the dew point is $30$ Celsius, make an equation that models the situation (1).

	      Hints:
	      \begin{enumerate}
		      \item{How can we find the dew point/what is it?}
		      \item{How would you write $1$ for every $5$ mathematically?}
	      \end{enumerate}

	      Answer: \(30=15+\frac{1}{5}R\)
	      }

	\item{Paper Scraper, Inc, a mobile paper shredding and recycling company, receives $\$0.39$ for each pound of paper they shred, but it costs them approximately $\$1.38$ per mile to operate their truck. Any job that requires less than $200$ miles of travel, and also nets at least $\$500$ profit, qualifies for free delivery. Of the following, which combination of size in pounds and distance in miles would qualify for free delivery?

		      \begin{enumerate}
			      \item {Paul has $1100$ pounds of paper that requires $3$ miles of travel}
			      \item {Li Min has $1500$ pounds of paper that requires $107$ miles of travel}
			      \item {Amira has $1800$ pounds of paper that requires $94$ miles of travel}
			      \item {Jarles has $3500$ pounds of paper that requires $225$ miles of travel}
		      \end{enumerate}

		      Hints:
		      \begin{enumerate}
			      \item{What is profit?}
			      \item{What linear equation could we come up with to model this situation?}
			      \item{The distance must be less than $200$}
		      \end{enumerate}

		      Answer: Use the linear equation \(500\leq 0.39p - 1.38m\) to see what job would be worth taking (A). You could solve for $p$, which might be easier.
	      }

	\item {A local grocer wants to mix candied pecans, priced at $\$14.00$ per pound and candied cashews priced at $\$10.00$ per pound. How many pounds of candied cashes must he mix with $8$ pounds of candied pecans to make a mixture that costs $\$12.50$ per pound? Round the answer to the nearest tenth of a pound (1).

		      Hints:
		      \begin{enumerate}
			      \item{Cost can be represented by the system of equations:

			            \[
				            \begin{array}{l}
					            \begin{cases}
						            c=10w+112 & \\
						            c=12.5(w+8)
					            \end{cases}
				            \end{array}
			            \]
			            }
			      \item {Set the equations equal to eachother and solve for $w$}
		      \end{enumerate}
		      Answer: $4.8$ lbs
	      }

	\item {People start waiting in line for the release of the newest cell phone at 5 a.m. The following equation gives the number of people $P$, in line between the hours, $h$, of 6 a.m. and 11 a.m., when the doors open. Assume that 6 a.m. is when time $h=1$. What does the $23$ mean in the equation (1)?

	      Answer: There are $23$ people in the line at 6 a.m.}

	\item {Mikayla is a waitress who makes a guaranteed $\$50$ per day in addition to tips of $20\%$ of all her weekly customer receipts, $t$. She works $6$ days per week. Write a function that models the amount of money that Mikayla makes in one week?

	      Hints:
	      \begin{enumerate}
		      \item{How much money does Mikayla make in a week, not regarding the $20\%$ of her weekly customer receipts?}
		      \item{Notice that the customer receipts, $t$, is referred to as the weekly customer receipts.}
	      \end{enumerate}

	      Answer: \(f(t)=300+0.2t\)}

	\item{A company manufactures cellular phones and laptop computers. The company's daily production of cellular phones, $c$, must be more than $500,000$, and its daily production of laptop computers, $l$, must be more than $300,000$. If the maximum capacity of the company's manufacturing center is no more than $950,00$ total cellular phones and laptop computers, create a system of inequalities that models the situation described.

	      Answer:
	      \[
		      \begin{array}{l}
			      \begin{cases}
				      c>500,000 & \\
				      l>300,000 & \\
				      c+l\leq950,000
			      \end{cases}
		      \end{array}
	      \]
	      }

\end{enumerate}

\section{Credits}
\begin{enumerate}

	\item{
	      \href{https://www.khanacademy.org/test-prep/v2-sat-math/x0fcc98a58ba3bea7:algebra-medium}{Khan Academy, Digital SAT Math, Unit 6}}

\end{enumerate}
\end{document}
